\documentclass{article}
\usepackage[utf8x]{inputenc}
\usepackage{amsmath}
\usepackage{amssymb}
\title{The Algebraic Structure of Integral Transforms and Friends}
\author{Chris Olah (\hbox{chris\@colah.ca})}

\let\mathbbm=\mathbb
\newcommand{\RFT}{\text{RFT}} 
\newcommand{\FT}{\text{FT}} 
\newcommand{\PT}{\text{P}} 
\newcommand{\PFT}{\text{PFT}} 
\newcommand{\Id}{\text{Id}} 
\renewcommand{\c}{\circ} 
\let\textalpha=\alpha
\let\textdelta=\delta
\let\textomega=\omega

\begin{document}

\maketitle

\section{Introduction}

Since I learned about them, I've been really interested in the ``big picture'' of integral transforms. What happens if you apply a Fourier Transform and then a Laplace Transform and then differentiate? Or similar combinations.

At first I thought category theory was the correct approach and I spent some time thinking about this. I didn't get anywhere, and I forgot about it for a while. Recently, I remembered the issue and started approaching it with group theory. I came up with some interesting results and would like to share them.

This seems like a really obvious topic to consider, and I'm certain work has been done. But I can't seem to find it,  likely because I am failing to use the right terms. If you are aware of anything, please contact me!

\section{The Fourier Transform}

The Fourier Transform is probably the most famous integral transform. It allows us to break a function into waves of different frequencies and shifts that build it up and its inverse allows us to reconstruct the original function from the frequencies, amplitudes and shifts of these waves.

One very surprising thing about this is that the Fourier Transform is very nearly its own inverse. Applying it twice gives you the original function back, just flipped.

$$\FT∘\FT(f)(x) = f(-x)$$

(That is: the function describing the frequencies of the waves that build up our original function, is built from waves described by our original function. That's pretty crazy!)

Now, since flipping a function twice takes one back to the original:

$$\FT⁴(f)(x) = f(-x)$$
$$\FT⁴ = \Id$$

Clearly, these form a group under composition, since composition is associative, there are inverses ($\FT^{-1} = \FT³$), and there is an identity element $\Id = \FT⁴$.

In particular, the group generated by $\FT$ is isomorphic to the integers modulo $4$:

$$<\!\FT\!> ~≅~ ℤ_4$$

\section{The fractional Fourier Transform}

Since the Fourier Transform generates a group isomorphic to $\mathbb{Z}_4$, you may have already started looking at it as a quarter turn rotation. This is the key insight behind the fractional Fourier transform.

If one conceptualizes the Fourier Transform as a quarter turn on the time-frequency plane. One can turn less than a quarter, getting the function as a mixture of time and frequency. This has applications, for example, in filtering out noise of a varying frequency or that is temporary.

Thus we can generate arbitrary cyclic groups. By choosing turns of $\frac{τ}{n}$ we can generate $ℤ_n$.

\section{Reciprocal Frequency Transform \& Friends}

The really interesting structures start to arise when we intentionally construct transforms to generate interesting groups. Here we construct a transform that switches integrals and derivatives, along with some interesting friends. We use this to build an infinite dihedral group. 

Let $\PT_n(f)$ be pointwise raising of $f$ to the $n$th power. That is, $ \PT_n(f)(x) = f(x)^n $. Then we define the $n$th power frequency transform as:

$$ \PFT_n = \FT^{-1} ∘ \PT_n ∘ \FT$$

$$ \PFT_n(f) = \FT^{-1}_x(FT(f)(x)^n)$$

($FT(f)(x)$ is the Fourier Transform of $f$ bound to the variable $x$. Then $\FT^{-1}_x$ is the inverse Fourier Transform with respect to $x$. The result of this is a function rather than a variable with respect to $x$.)

In particular, the reciprocal frequency transform, $\RFT$ is:

$$\RFT = \PFT_{-1} = \FT^{-1} ∘ \PT_{-1} ∘ \FT$$

$$\RFT(f)(x) = \FT^{-1}_x\left(\frac{1}{FT(f)(x)}\right)$$

That is, reciprocation in the frequency domain.

\subsection{Derivative Mapping}

$\PFT_n$ takes derivatives to the $n$th derivative. In particular, $\RFT$ takes derivatives to integrals and vice versa.

\begin{eqnarray*}
\PFT_n(Df) & = & \FT^{-1}_x(FT(Df)(x)^n)\\
& = & \FT^{-1}_x((ixFT(f)(x))^n)\\
& = & \FT^{-1}_x((ix)^nFT(f)(x))\\
& = & D^n\FT^{-1}_x(FT(f)(x))\\
& = & D^n\PFT_n(f)\\
\\
\RFT(Df) & = & \PFT_{-1}(Df)\\
& = & \int\PFT_{-1}(f)\\
& = & \int\RFT(f)\\
\end{eqnarray*}

\subsection{Inverse}

$\PFT_n$'s inverse is $\PFT_\frac{1}{n}$. In particular, $\RFT$ is its own inverse.

\begin{eqnarray*}
\PFT_n ∘ \PFT_\frac{1}{n} & = & (\FT^{-1} ∘  \PT_n ∘  \FT) ∘ (\FT^{-1}  ∘ \PT_\frac{1}{n}  ∘ \FT)\\
                          & = &  \FT^{-1} ∘  \PT_n ∘ (\FT  ∘  \FT^{-1}) ∘ \PT_\frac{1}{n}  ∘ \FT \\
                          & = &  \FT^{-1} ∘  \PT_n ∘                      \PT_\frac{1}{n}  ∘ \FT \\
                          & = &  \FT^{-1} ∘ (\PT_n ∘                      \PT_\frac{1}{n}) ∘ \FT \\
                          & = &  \FT^{-1} ∘                                                  \FT \\
                          & = &  \text{Id}\\
\end{eqnarray*}

\subsection{Fixed Points}

For example, consider the Dirac delta function, $\delta$,  is a fixed point of $\PFT_n$, for all $n$:

\begin{eqnarray*}
\PFT_n(δ) & = & \FT^{-1}_x( \FT(δ)(x)^n )\\
          & = & \FT^{-1}_x( 1^n         )\\
          & = & \FT^{-1}_x( 1           )\\
          & = &  δ                       \\
\end{eqnarray*}

\subsection{The Group $<D, \RFT>$}

Clearly $D$ can be considered to generate a group (isomorphic to $ℤ$) and $\RFT$ generates a group (isomorphic to $ℤ₂$). Together, they generate some quotient of the free product of these two. Since $\RFT ∘ D = D^{-1} ∘ \RFT$, we, in fact, get the infinite dihedral group $D_ω$.


\end{document}
